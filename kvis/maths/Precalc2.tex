\documentclass[a4paper,12pt]{article}
\usepackage[margin=2cm]{geometry}
\usepackage{amsmath}
\usepackage{multicol}
\newcommand{\im}{{i\mkern1mu}}

%opening
\title{Mathematics: Precalculus 2 Summary}
\author{Jirawut Thongraar}

\begin{document}

\maketitle

\section{Basic Formulas of Trigonometry}

\subsection{Pythagorean Identities}
$$\sin^2\theta+\cos^2\theta=1$$

\subsection{Sum and Difference Formulas}
$$\sin(\alpha+\beta)=\sin\alpha\cos\beta+\cos\alpha\sin\beta$$
$$\sin(\alpha-\beta)=\sin\alpha\cos\beta-\cos\alpha\sin\beta$$
$$\cos(\alpha+\beta)=\cos\alpha\cos\beta-\sin\alpha\sin\beta$$
$$\cos(\alpha-\beta)=\cos\alpha\cos\beta+\sin\alpha\sin\beta$$

\subsection{Cofunction Identities}
\begin{multicols}{2}
	$$\sin(\frac{\pi}{2}-\theta)=\cos\theta$$
	$$\tan(\frac{\pi}{2}-\theta)=\cot\theta$$
	$$\sec(\frac{\pi}{2}-\theta)=\csc\theta$$
	
	$$\cos(\frac{\pi}{2}-\theta)=\sin\theta$$
	$$\cot(\frac{\pi}{2}-\theta)=\tan\theta$$
	$$\csc(\frac{\pi}{2}-\theta)=\sec\theta$$
\end{multicols}

\subsection{Negative Angle Identities}
$$\sin(-\theta)=-\sin\theta$$
$$\cos(-\theta)=\cos\theta$$
$$\tan(-\theta)=-\tan\theta$$

\section{Derivation of Trigonometric Identities}

\subsection{Double Angle Formulas}
Apply the Sum and Difference Formulas with $\alpha=\beta=\theta$
$$\sin(2\theta)=2\sin\theta\cos\theta$$
$$\cos(2\theta)=\cos^2\theta-\sin^2\theta$$

\subsection{Sum and Difference Formulas of Tangent and Cotangent}
From tangent identity, we get
\begin{align*}
	\tan(\alpha+\beta)&=\frac{\sin(\alpha+\beta)}{\cos(\alpha+\beta)} \\
	&=\frac{\sin\alpha\cos\beta+\cos\alpha\sin\beta}{\cos\alpha\cos\beta-\sin\alpha\sin\beta}
	\intertext{Dividing both numerator and denominator by $\cos\alpha\cos\beta$, we get}
	\tan(\alpha+\beta)&=\frac{\tan\alpha+\tan\beta}{1-\tan\alpha\tan\beta}
\end{align*}
Similar process can be done to obtain $\cot(\alpha+\beta)$. $\sin\alpha\sin\beta$ is used to divide numerator and denominator instead.

\subsection{Multiple Angle Formulas}
Derivation of Multiple Angle Formulas utilize lower multiple angle formulas.
$$\sin3\theta=\sin(2\theta+\theta)=3\sin\theta-4\sin^3\theta$$
$$\cos3\theta=\cos(2\theta+\theta)=4\cos^3\theta-3\cos\theta$$
This process can be done for higher multiple angle formulas.

\subsection{Power Reducing Formulas}
Take the double angle cosine formula
$$\cos(2\theta)=\cos^2\theta-\sin^2\theta$$
From Pythagorean Identities, substitute $\sin^2\theta$ and $\cos^2\theta$, we get
\begin{multicols}{2}
	$$\cos(2\theta)=\cos^2\theta-(1-\cos^2\theta)$$
	$$\cos^2\theta=\frac{1+\cos(2\theta)}{2}$$
	
	$$\cos(2\theta)=(1-\sin^2\theta)-\sin^2\theta$$
	$$\sin^2\theta=\frac{1-\cos(2\theta)}{2}$$
\end{multicols}

\subsection{Half Angle Formulas}
Use Power Reducing Function on $\frac{\theta}{2}$, we get
\begin{multicols}{2}
	$$\sin^2\frac{\theta}{2}=\frac{1-\cos2(\frac{\theta}{2})}{2}$$
	$$\sin\frac{\theta}{2}=\pm\sqrt{\frac{1-\cos2(\frac{\theta}{2})}{2}}$$
	
	$$\cos^2\frac{\theta}{2}=\frac{1+\cos2(\frac{\theta}{2})}{2}$$
	$$\cos\frac{\theta}{2}=\pm\sqrt{\frac{1+\cos2(\frac{\theta}{2})}{2}}$$
\end{multicols}
Signs depended on the quadrant of $\frac{\theta}{2}$, which determines the sign of $\sin\frac{\theta}{2}$ and $\cos\frac{\theta}{2}$.

\subsection{Product to Sum Formulas}
These class of formulas can be derived via summation of Sum and Difference Formulas.
\begin{align*}
	\sin(\alpha+\beta)-\sin(\alpha-\beta)&=(\sin\alpha\cos\beta+\cos\alpha\sin\beta)-(\sin\alpha\cos\beta-\cos\alpha\sin\beta) \\
	&=2\cos\alpha\sin\beta
\end{align*}
Similar process can be done to get other product to sum formulas.

\subsection{Sum to Product Formulas}
From $\sin(\alpha+\beta)-\sin(\alpha-\beta)=2\sin\alpha\cos\beta$, Let $\alpha+\beta=u$, $\alpha-\beta=v$, we get
\begin{align*}
	\sin(u)-\sin(v)&=\sin(\alpha+\beta)-\sin(\alpha-\beta) \\
	&=2\cos(\frac{u+v}{2})\sin(\frac{u-v}{2})
\end{align*}
Similar process can be done to get other sum to product formulas.


\section{Application of Trigonometries in Triangle}
Let $ABC$ be triangle where $a, b, c$ are sides.\\
Law of Sines.
$$\frac{a}{\sin A} =\frac{b}{\sin B}=\frac{c}{\sin C} $$
$$Area=\frac{1}{2}(base)(height)=\frac{1}{2}bc\sin A=\frac{1}{2}ac\sin B=\frac{1}{2}ab\sin C$$
Law of Cosines.
$$a^2=b^2+c^2-2bc\cos A$$
$$b^2=a^2+c^2-2ac\cos B$$
$$c^2=a^2+b^2-2ab\cos C$$
Heron's Formula
$$Area=\sqrt{s(s-a)(s-b)(s-c)}$$ where $s=\frac{a+b+c}{2}$


\section{Application of Trigonometries in Complex Number}
Let $z$ represents a complex number $a+b\im$
$$z=a+b\im=(|z|cos\theta)+(|z|sin\theta)\im$$
Multiplication and Division
$$z_1z_2=|z_1||z_2|(cos(\theta_1+\theta_2)+sin(\theta_1+\theta_2)\im)$$
$$\frac{z_1}{z_2}=\frac{|z_1|}{|z_2|}(cos(\theta_1-\theta_2)+sin(\theta_1-\theta_2)\im)$$
Power and Root
$$z^n=|z|^n(\cos n\theta+\sin n\theta \im)$$
$$\sqrt[n]{z}=\sqrt[n]{|z|}(cos\frac{\theta+2\pi k}{n}+\sin(\frac{\theta+2\pi k}{n})\im)$$ for $k=0,1,...,n-1$


\section{Vectors (3 Dimension)}
Distance between $P_1=(x_1,y_1,z_2)$ and $P_2=(x_2,y_2,z_2)$ is
$$\sqrt{(x_1-x_2)^2+(y_1-y_2)^2+(z_1-z_2)^2}$$

\subsection{Dot Product}
Let $a=\langle a_1,a_2,a_3\rangle$ and $b=\langle b_1,b_2,b_3\rangle$
$$a\cdot b=a_1b_1+a_2b_2+a_3b_3=|a||b|\cos\theta$$
We can project a vector $v$ on to direction of unit vector $e$ as
$$\text{proj}_e v=(v\cdot e)e$$

\subsection{Cross Product}
Let $a=\langle a_1,a_2,a_3\rangle$ and $b=\langle b_1,b_2,b_3\rangle$
$$a\times b=\begin{vmatrix}
 i & j & k \\
 a_1 & a_2 & a_3 \\
 b_1 & b_2 & b_3
\end{vmatrix}$$
\\
Let $a=\langle a_1,a_2,a_3\rangle$, $b=\langle b_1,b_2,b_3\rangle$, and $c=\langle c_1,c_2,c_3\rangle$
$$a\cdot(b\times c)=\begin{vmatrix}
a_1 & a_2 & a_3 \\
b_1 & b_2 & b_3 \\
c_1 & c_2 & c_3
\end{vmatrix}$$
The above value is also the volume of parallel shape formed by vector $a, b, c$

\end{document}
